
% Default to the notebook output style

    


% Inherit from the specified cell style.




    
\documentclass[11pt]{article}

    
    
    \usepackage[T1]{fontenc}
    % Nicer default font (+ math font) than Computer Modern for most use cases
    \usepackage{mathpazo}

    % Basic figure setup, for now with no caption control since it's done
    % automatically by Pandoc (which extracts ![](path) syntax from Markdown).
    \usepackage{graphicx}
    % We will generate all images so they have a width \maxwidth. This means
    % that they will get their normal width if they fit onto the page, but
    % are scaled down if they would overflow the margins.
    \makeatletter
    \def\maxwidth{\ifdim\Gin@nat@width>\linewidth\linewidth
    \else\Gin@nat@width\fi}
    \makeatother
    \let\Oldincludegraphics\includegraphics
    % Set max figure width to be 80% of text width, for now hardcoded.
    \renewcommand{\includegraphics}[1]{\Oldincludegraphics[width=.8\maxwidth]{#1}}
    % Ensure that by default, figures have no caption (until we provide a
    % proper Figure object with a Caption API and a way to capture that
    % in the conversion process - todo).
    \usepackage{caption}
    \DeclareCaptionLabelFormat{nolabel}{}
    \captionsetup{labelformat=nolabel}

    \usepackage{adjustbox} % Used to constrain images to a maximum size 
    \usepackage{xcolor} % Allow colors to be defined
    \usepackage{enumerate} % Needed for markdown enumerations to work
    \usepackage{geometry} % Used to adjust the document margins
    \usepackage{amsmath} % Equations
    \usepackage{amssymb} % Equations
    \usepackage{textcomp} % defines textquotesingle
    % Hack from http://tex.stackexchange.com/a/47451/13684:
    \AtBeginDocument{%
        \def\PYZsq{\textquotesingle}% Upright quotes in Pygmentized code
    }
    \usepackage{upquote} % Upright quotes for verbatim code
    \usepackage{eurosym} % defines \euro
    \usepackage[mathletters]{ucs} % Extended unicode (utf-8) support
    \usepackage[utf8x]{inputenc} % Allow utf-8 characters in the tex document
    \usepackage{fancyvrb} % verbatim replacement that allows latex
    \usepackage{grffile} % extends the file name processing of package graphics 
                         % to support a larger range 
    % The hyperref package gives us a pdf with properly built
    % internal navigation ('pdf bookmarks' for the table of contents,
    % internal cross-reference links, web links for URLs, etc.)
    \usepackage{hyperref}
    \usepackage{longtable} % longtable support required by pandoc >1.10
    \usepackage{booktabs}  % table support for pandoc > 1.12.2
    \usepackage[inline]{enumitem} % IRkernel/repr support (it uses the enumerate* environment)
    \usepackage[normalem]{ulem} % ulem is needed to support strikethroughs (\sout)
                                % normalem makes italics be italics, not underlines
    

    
    
    % Colors for the hyperref package
    \definecolor{urlcolor}{rgb}{0,.145,.698}
    \definecolor{linkcolor}{rgb}{.71,0.21,0.01}
    \definecolor{citecolor}{rgb}{.12,.54,.11}

    % ANSI colors
    \definecolor{ansi-black}{HTML}{3E424D}
    \definecolor{ansi-black-intense}{HTML}{282C36}
    \definecolor{ansi-red}{HTML}{E75C58}
    \definecolor{ansi-red-intense}{HTML}{B22B31}
    \definecolor{ansi-green}{HTML}{00A250}
    \definecolor{ansi-green-intense}{HTML}{007427}
    \definecolor{ansi-yellow}{HTML}{DDB62B}
    \definecolor{ansi-yellow-intense}{HTML}{B27D12}
    \definecolor{ansi-blue}{HTML}{208FFB}
    \definecolor{ansi-blue-intense}{HTML}{0065CA}
    \definecolor{ansi-magenta}{HTML}{D160C4}
    \definecolor{ansi-magenta-intense}{HTML}{A03196}
    \definecolor{ansi-cyan}{HTML}{60C6C8}
    \definecolor{ansi-cyan-intense}{HTML}{258F8F}
    \definecolor{ansi-white}{HTML}{C5C1B4}
    \definecolor{ansi-white-intense}{HTML}{A1A6B2}

    % commands and environments needed by pandoc snippets
    % extracted from the output of `pandoc -s`
    \providecommand{\tightlist}{%
      \setlength{\itemsep}{0pt}\setlength{\parskip}{0pt}}
    \DefineVerbatimEnvironment{Highlighting}{Verbatim}{commandchars=\\\{\}}
    % Add ',fontsize=\small' for more characters per line
    \newenvironment{Shaded}{}{}
    \newcommand{\KeywordTok}[1]{\textcolor[rgb]{0.00,0.44,0.13}{\textbf{{#1}}}}
    \newcommand{\DataTypeTok}[1]{\textcolor[rgb]{0.56,0.13,0.00}{{#1}}}
    \newcommand{\DecValTok}[1]{\textcolor[rgb]{0.25,0.63,0.44}{{#1}}}
    \newcommand{\BaseNTok}[1]{\textcolor[rgb]{0.25,0.63,0.44}{{#1}}}
    \newcommand{\FloatTok}[1]{\textcolor[rgb]{0.25,0.63,0.44}{{#1}}}
    \newcommand{\CharTok}[1]{\textcolor[rgb]{0.25,0.44,0.63}{{#1}}}
    \newcommand{\StringTok}[1]{\textcolor[rgb]{0.25,0.44,0.63}{{#1}}}
    \newcommand{\CommentTok}[1]{\textcolor[rgb]{0.38,0.63,0.69}{\textit{{#1}}}}
    \newcommand{\OtherTok}[1]{\textcolor[rgb]{0.00,0.44,0.13}{{#1}}}
    \newcommand{\AlertTok}[1]{\textcolor[rgb]{1.00,0.00,0.00}{\textbf{{#1}}}}
    \newcommand{\FunctionTok}[1]{\textcolor[rgb]{0.02,0.16,0.49}{{#1}}}
    \newcommand{\RegionMarkerTok}[1]{{#1}}
    \newcommand{\ErrorTok}[1]{\textcolor[rgb]{1.00,0.00,0.00}{\textbf{{#1}}}}
    \newcommand{\NormalTok}[1]{{#1}}
    
    % Additional commands for more recent versions of Pandoc
    \newcommand{\ConstantTok}[1]{\textcolor[rgb]{0.53,0.00,0.00}{{#1}}}
    \newcommand{\SpecialCharTok}[1]{\textcolor[rgb]{0.25,0.44,0.63}{{#1}}}
    \newcommand{\VerbatimStringTok}[1]{\textcolor[rgb]{0.25,0.44,0.63}{{#1}}}
    \newcommand{\SpecialStringTok}[1]{\textcolor[rgb]{0.73,0.40,0.53}{{#1}}}
    \newcommand{\ImportTok}[1]{{#1}}
    \newcommand{\DocumentationTok}[1]{\textcolor[rgb]{0.73,0.13,0.13}{\textit{{#1}}}}
    \newcommand{\AnnotationTok}[1]{\textcolor[rgb]{0.38,0.63,0.69}{\textbf{\textit{{#1}}}}}
    \newcommand{\CommentVarTok}[1]{\textcolor[rgb]{0.38,0.63,0.69}{\textbf{\textit{{#1}}}}}
    \newcommand{\VariableTok}[1]{\textcolor[rgb]{0.10,0.09,0.49}{{#1}}}
    \newcommand{\ControlFlowTok}[1]{\textcolor[rgb]{0.00,0.44,0.13}{\textbf{{#1}}}}
    \newcommand{\OperatorTok}[1]{\textcolor[rgb]{0.40,0.40,0.40}{{#1}}}
    \newcommand{\BuiltInTok}[1]{{#1}}
    \newcommand{\ExtensionTok}[1]{{#1}}
    \newcommand{\PreprocessorTok}[1]{\textcolor[rgb]{0.74,0.48,0.00}{{#1}}}
    \newcommand{\AttributeTok}[1]{\textcolor[rgb]{0.49,0.56,0.16}{{#1}}}
    \newcommand{\InformationTok}[1]{\textcolor[rgb]{0.38,0.63,0.69}{\textbf{\textit{{#1}}}}}
    \newcommand{\WarningTok}[1]{\textcolor[rgb]{0.38,0.63,0.69}{\textbf{\textit{{#1}}}}}
    
    
    % Define a nice break command that doesn't care if a line doesn't already
    % exist.
    \def\br{\hspace*{\fill} \\* }
    % Math Jax compatability definitions
    \def\gt{>}
    \def\lt{<}
    % Document parameters
    \title{Homework 3}
    
    
    

    % Pygments definitions
    
\makeatletter
\def\PY@reset{\let\PY@it=\relax \let\PY@bf=\relax%
    \let\PY@ul=\relax \let\PY@tc=\relax%
    \let\PY@bc=\relax \let\PY@ff=\relax}
\def\PY@tok#1{\csname PY@tok@#1\endcsname}
\def\PY@toks#1+{\ifx\relax#1\empty\else%
    \PY@tok{#1}\expandafter\PY@toks\fi}
\def\PY@do#1{\PY@bc{\PY@tc{\PY@ul{%
    \PY@it{\PY@bf{\PY@ff{#1}}}}}}}
\def\PY#1#2{\PY@reset\PY@toks#1+\relax+\PY@do{#2}}

\expandafter\def\csname PY@tok@w\endcsname{\def\PY@tc##1{\textcolor[rgb]{0.73,0.73,0.73}{##1}}}
\expandafter\def\csname PY@tok@c\endcsname{\let\PY@it=\textit\def\PY@tc##1{\textcolor[rgb]{0.25,0.50,0.50}{##1}}}
\expandafter\def\csname PY@tok@cp\endcsname{\def\PY@tc##1{\textcolor[rgb]{0.74,0.48,0.00}{##1}}}
\expandafter\def\csname PY@tok@k\endcsname{\let\PY@bf=\textbf\def\PY@tc##1{\textcolor[rgb]{0.00,0.50,0.00}{##1}}}
\expandafter\def\csname PY@tok@kp\endcsname{\def\PY@tc##1{\textcolor[rgb]{0.00,0.50,0.00}{##1}}}
\expandafter\def\csname PY@tok@kt\endcsname{\def\PY@tc##1{\textcolor[rgb]{0.69,0.00,0.25}{##1}}}
\expandafter\def\csname PY@tok@o\endcsname{\def\PY@tc##1{\textcolor[rgb]{0.40,0.40,0.40}{##1}}}
\expandafter\def\csname PY@tok@ow\endcsname{\let\PY@bf=\textbf\def\PY@tc##1{\textcolor[rgb]{0.67,0.13,1.00}{##1}}}
\expandafter\def\csname PY@tok@nb\endcsname{\def\PY@tc##1{\textcolor[rgb]{0.00,0.50,0.00}{##1}}}
\expandafter\def\csname PY@tok@nf\endcsname{\def\PY@tc##1{\textcolor[rgb]{0.00,0.00,1.00}{##1}}}
\expandafter\def\csname PY@tok@nc\endcsname{\let\PY@bf=\textbf\def\PY@tc##1{\textcolor[rgb]{0.00,0.00,1.00}{##1}}}
\expandafter\def\csname PY@tok@nn\endcsname{\let\PY@bf=\textbf\def\PY@tc##1{\textcolor[rgb]{0.00,0.00,1.00}{##1}}}
\expandafter\def\csname PY@tok@ne\endcsname{\let\PY@bf=\textbf\def\PY@tc##1{\textcolor[rgb]{0.82,0.25,0.23}{##1}}}
\expandafter\def\csname PY@tok@nv\endcsname{\def\PY@tc##1{\textcolor[rgb]{0.10,0.09,0.49}{##1}}}
\expandafter\def\csname PY@tok@no\endcsname{\def\PY@tc##1{\textcolor[rgb]{0.53,0.00,0.00}{##1}}}
\expandafter\def\csname PY@tok@nl\endcsname{\def\PY@tc##1{\textcolor[rgb]{0.63,0.63,0.00}{##1}}}
\expandafter\def\csname PY@tok@ni\endcsname{\let\PY@bf=\textbf\def\PY@tc##1{\textcolor[rgb]{0.60,0.60,0.60}{##1}}}
\expandafter\def\csname PY@tok@na\endcsname{\def\PY@tc##1{\textcolor[rgb]{0.49,0.56,0.16}{##1}}}
\expandafter\def\csname PY@tok@nt\endcsname{\let\PY@bf=\textbf\def\PY@tc##1{\textcolor[rgb]{0.00,0.50,0.00}{##1}}}
\expandafter\def\csname PY@tok@nd\endcsname{\def\PY@tc##1{\textcolor[rgb]{0.67,0.13,1.00}{##1}}}
\expandafter\def\csname PY@tok@s\endcsname{\def\PY@tc##1{\textcolor[rgb]{0.73,0.13,0.13}{##1}}}
\expandafter\def\csname PY@tok@sd\endcsname{\let\PY@it=\textit\def\PY@tc##1{\textcolor[rgb]{0.73,0.13,0.13}{##1}}}
\expandafter\def\csname PY@tok@si\endcsname{\let\PY@bf=\textbf\def\PY@tc##1{\textcolor[rgb]{0.73,0.40,0.53}{##1}}}
\expandafter\def\csname PY@tok@se\endcsname{\let\PY@bf=\textbf\def\PY@tc##1{\textcolor[rgb]{0.73,0.40,0.13}{##1}}}
\expandafter\def\csname PY@tok@sr\endcsname{\def\PY@tc##1{\textcolor[rgb]{0.73,0.40,0.53}{##1}}}
\expandafter\def\csname PY@tok@ss\endcsname{\def\PY@tc##1{\textcolor[rgb]{0.10,0.09,0.49}{##1}}}
\expandafter\def\csname PY@tok@sx\endcsname{\def\PY@tc##1{\textcolor[rgb]{0.00,0.50,0.00}{##1}}}
\expandafter\def\csname PY@tok@m\endcsname{\def\PY@tc##1{\textcolor[rgb]{0.40,0.40,0.40}{##1}}}
\expandafter\def\csname PY@tok@gh\endcsname{\let\PY@bf=\textbf\def\PY@tc##1{\textcolor[rgb]{0.00,0.00,0.50}{##1}}}
\expandafter\def\csname PY@tok@gu\endcsname{\let\PY@bf=\textbf\def\PY@tc##1{\textcolor[rgb]{0.50,0.00,0.50}{##1}}}
\expandafter\def\csname PY@tok@gd\endcsname{\def\PY@tc##1{\textcolor[rgb]{0.63,0.00,0.00}{##1}}}
\expandafter\def\csname PY@tok@gi\endcsname{\def\PY@tc##1{\textcolor[rgb]{0.00,0.63,0.00}{##1}}}
\expandafter\def\csname PY@tok@gr\endcsname{\def\PY@tc##1{\textcolor[rgb]{1.00,0.00,0.00}{##1}}}
\expandafter\def\csname PY@tok@ge\endcsname{\let\PY@it=\textit}
\expandafter\def\csname PY@tok@gs\endcsname{\let\PY@bf=\textbf}
\expandafter\def\csname PY@tok@gp\endcsname{\let\PY@bf=\textbf\def\PY@tc##1{\textcolor[rgb]{0.00,0.00,0.50}{##1}}}
\expandafter\def\csname PY@tok@go\endcsname{\def\PY@tc##1{\textcolor[rgb]{0.53,0.53,0.53}{##1}}}
\expandafter\def\csname PY@tok@gt\endcsname{\def\PY@tc##1{\textcolor[rgb]{0.00,0.27,0.87}{##1}}}
\expandafter\def\csname PY@tok@err\endcsname{\def\PY@bc##1{\setlength{\fboxsep}{0pt}\fcolorbox[rgb]{1.00,0.00,0.00}{1,1,1}{\strut ##1}}}
\expandafter\def\csname PY@tok@kc\endcsname{\let\PY@bf=\textbf\def\PY@tc##1{\textcolor[rgb]{0.00,0.50,0.00}{##1}}}
\expandafter\def\csname PY@tok@kd\endcsname{\let\PY@bf=\textbf\def\PY@tc##1{\textcolor[rgb]{0.00,0.50,0.00}{##1}}}
\expandafter\def\csname PY@tok@kn\endcsname{\let\PY@bf=\textbf\def\PY@tc##1{\textcolor[rgb]{0.00,0.50,0.00}{##1}}}
\expandafter\def\csname PY@tok@kr\endcsname{\let\PY@bf=\textbf\def\PY@tc##1{\textcolor[rgb]{0.00,0.50,0.00}{##1}}}
\expandafter\def\csname PY@tok@bp\endcsname{\def\PY@tc##1{\textcolor[rgb]{0.00,0.50,0.00}{##1}}}
\expandafter\def\csname PY@tok@fm\endcsname{\def\PY@tc##1{\textcolor[rgb]{0.00,0.00,1.00}{##1}}}
\expandafter\def\csname PY@tok@vc\endcsname{\def\PY@tc##1{\textcolor[rgb]{0.10,0.09,0.49}{##1}}}
\expandafter\def\csname PY@tok@vg\endcsname{\def\PY@tc##1{\textcolor[rgb]{0.10,0.09,0.49}{##1}}}
\expandafter\def\csname PY@tok@vi\endcsname{\def\PY@tc##1{\textcolor[rgb]{0.10,0.09,0.49}{##1}}}
\expandafter\def\csname PY@tok@vm\endcsname{\def\PY@tc##1{\textcolor[rgb]{0.10,0.09,0.49}{##1}}}
\expandafter\def\csname PY@tok@sa\endcsname{\def\PY@tc##1{\textcolor[rgb]{0.73,0.13,0.13}{##1}}}
\expandafter\def\csname PY@tok@sb\endcsname{\def\PY@tc##1{\textcolor[rgb]{0.73,0.13,0.13}{##1}}}
\expandafter\def\csname PY@tok@sc\endcsname{\def\PY@tc##1{\textcolor[rgb]{0.73,0.13,0.13}{##1}}}
\expandafter\def\csname PY@tok@dl\endcsname{\def\PY@tc##1{\textcolor[rgb]{0.73,0.13,0.13}{##1}}}
\expandafter\def\csname PY@tok@s2\endcsname{\def\PY@tc##1{\textcolor[rgb]{0.73,0.13,0.13}{##1}}}
\expandafter\def\csname PY@tok@sh\endcsname{\def\PY@tc##1{\textcolor[rgb]{0.73,0.13,0.13}{##1}}}
\expandafter\def\csname PY@tok@s1\endcsname{\def\PY@tc##1{\textcolor[rgb]{0.73,0.13,0.13}{##1}}}
\expandafter\def\csname PY@tok@mb\endcsname{\def\PY@tc##1{\textcolor[rgb]{0.40,0.40,0.40}{##1}}}
\expandafter\def\csname PY@tok@mf\endcsname{\def\PY@tc##1{\textcolor[rgb]{0.40,0.40,0.40}{##1}}}
\expandafter\def\csname PY@tok@mh\endcsname{\def\PY@tc##1{\textcolor[rgb]{0.40,0.40,0.40}{##1}}}
\expandafter\def\csname PY@tok@mi\endcsname{\def\PY@tc##1{\textcolor[rgb]{0.40,0.40,0.40}{##1}}}
\expandafter\def\csname PY@tok@il\endcsname{\def\PY@tc##1{\textcolor[rgb]{0.40,0.40,0.40}{##1}}}
\expandafter\def\csname PY@tok@mo\endcsname{\def\PY@tc##1{\textcolor[rgb]{0.40,0.40,0.40}{##1}}}
\expandafter\def\csname PY@tok@ch\endcsname{\let\PY@it=\textit\def\PY@tc##1{\textcolor[rgb]{0.25,0.50,0.50}{##1}}}
\expandafter\def\csname PY@tok@cm\endcsname{\let\PY@it=\textit\def\PY@tc##1{\textcolor[rgb]{0.25,0.50,0.50}{##1}}}
\expandafter\def\csname PY@tok@cpf\endcsname{\let\PY@it=\textit\def\PY@tc##1{\textcolor[rgb]{0.25,0.50,0.50}{##1}}}
\expandafter\def\csname PY@tok@c1\endcsname{\let\PY@it=\textit\def\PY@tc##1{\textcolor[rgb]{0.25,0.50,0.50}{##1}}}
\expandafter\def\csname PY@tok@cs\endcsname{\let\PY@it=\textit\def\PY@tc##1{\textcolor[rgb]{0.25,0.50,0.50}{##1}}}

\def\PYZbs{\char`\\}
\def\PYZus{\char`\_}
\def\PYZob{\char`\{}
\def\PYZcb{\char`\}}
\def\PYZca{\char`\^}
\def\PYZam{\char`\&}
\def\PYZlt{\char`\<}
\def\PYZgt{\char`\>}
\def\PYZsh{\char`\#}
\def\PYZpc{\char`\%}
\def\PYZdl{\char`\$}
\def\PYZhy{\char`\-}
\def\PYZsq{\char`\'}
\def\PYZdq{\char`\"}
\def\PYZti{\char`\~}
% for compatibility with earlier versions
\def\PYZat{@}
\def\PYZlb{[}
\def\PYZrb{]}
\makeatother


    % Exact colors from NB
    \definecolor{incolor}{rgb}{0.0, 0.0, 0.5}
    \definecolor{outcolor}{rgb}{0.545, 0.0, 0.0}



    
    % Prevent overflowing lines due to hard-to-break entities
    \sloppy 
    % Setup hyperref package
    \hypersetup{
      breaklinks=true,  % so long urls are correctly broken across lines
      colorlinks=true,
      urlcolor=urlcolor,
      linkcolor=linkcolor,
      citecolor=citecolor,
      }
    % Slightly bigger margins than the latex defaults
    
    \geometry{verbose,tmargin=1in,bmargin=1in,lmargin=1in,rmargin=1in}
    
    

    \begin{document}
    
    
    \maketitle
    
    

    
    \section{Homework 3}\label{homework-3}

\subsection{Theory Part}\label{theory-part}

    \textbf{Q1. Consider a CNN that has} 1. Input of 14x14 with 30 channels.
2. A convolution layer C with 12 filters, each of size 4x4. The
convolution zero padding is 1 and stride is 2, followed by a ReLU
activation. 3. A max pooling layer P that is applied over each of the
C's output feature map, using 3x3 receptive field and stride 2.

\textbf{What is the total size of C's output feature map?}

Convolution output size is given as

\[ O_{size} = ceil(( M + 2r - ksize + 1) / k) \]

where \(M\) is the size of the concerned dimension, \(r\) is padding and
\(k\) is stride, and \(ksize\) is kernel size of the corresponding
dimension. Therefore,

\[ O_{size} = ceil(( 14 + 2*1 - 4 + 1) / 2) \]

    \begin{Verbatim}[commandchars=\\\{\}]
{\color{incolor}In [{\color{incolor}1}]:} \PY{k+kn}{import} \PY{n+nn}{math}
        
        \PY{n}{Osize} \PY{o}{=} \PY{n}{math}\PY{o}{.}\PY{n}{ceil}\PY{p}{(}\PY{p}{(}\PY{l+m+mi}{14} \PY{o}{+} \PY{l+m+mi}{2}\PY{o}{*}\PY{l+m+mi}{1} \PY{o}{\PYZhy{}} \PY{l+m+mi}{4} \PY{o}{+} \PY{l+m+mi}{1}\PY{p}{)}\PY{o}{/}\PY{l+m+mi}{2}\PY{p}{)}
        \PY{n+nb}{print}\PY{p}{(}\PY{n}{Osize}\PY{p}{)}
\end{Verbatim}


    \begin{Verbatim}[commandchars=\\\{\}]
7

    \end{Verbatim}

    So the total size of C's output feature map is 7x7.

\textbf{What is the total size of P's output feature map?}

Same formula, just without the padding.

    \begin{Verbatim}[commandchars=\\\{\}]
{\color{incolor}In [{\color{incolor}2}]:} \PY{n}{Osize} \PY{o}{=} \PY{n}{math}\PY{o}{.}\PY{n}{ceil}\PY{p}{(}\PY{p}{(}\PY{l+m+mi}{7} \PY{o}{\PYZhy{}} \PY{l+m+mi}{3} \PY{o}{+} \PY{l+m+mi}{1}\PY{p}{)}\PY{o}{/}\PY{l+m+mi}{2}\PY{p}{)}
        \PY{n+nb}{print}\PY{p}{(}\PY{n}{Osize}\PY{p}{)}
\end{Verbatim}


    \begin{Verbatim}[commandchars=\\\{\}]
3

    \end{Verbatim}

    So the total size of O's output feature map is 3x3.

\begin{center}\rule{0.5\linewidth}{\linethickness}\end{center}

    Now we want to compute the overhead of the above CNN in terms of
floating point operation (FLOP). FLOP can be used to measure computer's
performance. A decent processor nowadays can perform in Giga-FLOPS, that
means billions of FLOP per second. Assume the inputs are all scalars (we
have 14 × 14 × 30 scalars as input), we have the computational cost of:
1. 1 FLOP for a single scalar multiplication xi · xj 2. 1 FLOP for a
single scalar addition xi + xj 3. (n − 1) FLOPs for a max operation over
n items: max\{x1, ..., xn\}

\textbf{How many FLOPs layer C and P cost in total to do one forward
pass?}

    For each cell of each filter in C's output feature map, there must
necessarily be \(ksize^2\) number of \emph{multiplication} done for each
of the input \emph{channel} (for this 2d kernel).

\[ C_{mul} = O_{size}^2 * channels * ksize^2 * filters\]
\[ C_{mul} = 7^2 * 30 * 4^2 * 12\] \[ C_{mul} = 282240\]

    \begin{Verbatim}[commandchars=\\\{\}]
{\color{incolor}In [{\color{incolor}3}]:} \PY{l+m+mi}{7}\PY{o}{*}\PY{o}{*}\PY{l+m+mi}{2} \PY{o}{*} \PY{l+m+mi}{30} \PY{o}{*} \PY{l+m+mi}{4}\PY{o}{*}\PY{o}{*}\PY{l+m+mi}{2} \PY{o}{*} \PY{l+m+mi}{12}
\end{Verbatim}


\begin{Verbatim}[commandchars=\\\{\}]
{\color{outcolor}Out[{\color{outcolor}3}]:} 282240
\end{Verbatim}
            
    For each cell of each filter in C's output feature map, there must also
necessarily be \((ksize^2 - 1)\) number of \emph{addition} (for this 2d
kernel) to sum up all multiplication that is done for that output cell,
for each of the input \emph{channel} .

\[ C_{add} = O_{size}^2 * (ksize^2 - 1) * filters * channel\]
\[ C_{add} = 7^2 * (4^2*30 - 1) * 12\] \[ C_{add} = 281652\]

    \begin{Verbatim}[commandchars=\\\{\}]
{\color{incolor}In [{\color{incolor}4}]:} \PY{l+m+mi}{7}\PY{o}{*}\PY{o}{*}\PY{l+m+mi}{2} \PY{o}{*} \PY{p}{(}\PY{l+m+mi}{4}\PY{o}{*}\PY{o}{*}\PY{l+m+mi}{2} \PY{o}{*} \PY{l+m+mi}{30} \PY{o}{\PYZhy{}} \PY{l+m+mi}{1}\PY{p}{)} \PY{o}{*} \PY{l+m+mi}{12}
\end{Verbatim}


\begin{Verbatim}[commandchars=\\\{\}]
{\color{outcolor}Out[{\color{outcolor}4}]:} 281652
\end{Verbatim}
            
    For each cell of P's output feature map, there must be a \emph{max
operation} operating on \(ksize^2\) cells of the previous map.

\[ P_{max} = O_{size}^2 * C_{channel} * (ksize^2 - 1) \]

\[ P_{max} = 3^2 * 12 * (3^2 - 1)\]

\[ P_{max} = 864\]

    \begin{Verbatim}[commandchars=\\\{\}]
{\color{incolor}In [{\color{incolor}5}]:} \PY{l+m+mi}{3}\PY{o}{*}\PY{o}{*}\PY{l+m+mi}{2} \PY{o}{*} \PY{l+m+mi}{12} \PY{o}{*} \PY{p}{(}\PY{l+m+mi}{3}\PY{o}{*}\PY{o}{*}\PY{l+m+mi}{2} \PY{o}{\PYZhy{}}\PY{l+m+mi}{1}\PY{p}{)}
\end{Verbatim}


\begin{Verbatim}[commandchars=\\\{\}]
{\color{outcolor}Out[{\color{outcolor}5}]:} 864
\end{Verbatim}
            
    so the total FLOP is,

\[ T_{FLOP} = C_{mul} + C_{add} + P_{max} \]

\[ T_{FLOP} = 282240 + 281652 + 864 \]

\[ T_{FLOP} = 564756 \]

    \begin{Verbatim}[commandchars=\\\{\}]
{\color{incolor}In [{\color{incolor}6}]:} \PY{l+m+mi}{282240} \PY{o}{+} \PY{l+m+mi}{281652} \PY{o}{+} \PY{l+m+mi}{864}
\end{Verbatim}


\begin{Verbatim}[commandchars=\\\{\}]
{\color{outcolor}Out[{\color{outcolor}6}]:} 564756
\end{Verbatim}
            
    \begin{center}\rule{0.5\linewidth}{\linethickness}\end{center}

..

\textbf{Q2 Refer to the neural network at \emph{figure 1} with input
\(x ∈ R^1\). The activation function for \(z_1\), \(z_2\), and \(z_3\)
is the sigmoid function: \(\frac{1}{1+e^{-w \cdot x}}\),}

\[ h(x) = \frac{1}{1+e^{-x}} \quad\quad (1)\]

\[ z_1 = h(x \cdot w_{(x,1)}) \quad\quad (2)\]

\[ z_2 = h(z_1 \cdot w_{(1,2)}) \quad\quad (3)\]

\[ z_3 = h(z_1 \cdot w_{(1,3)}) \quad\quad (4)\]

For the error E, instead of using the softmax function we learned in
class, we use the quadratic error function for regression purpose,

\[ E = \sum_{i \space \epsilon \space data } ((z_2 − y_{2i})^2 + (z_3 − y_{3i})^2)\]

{[}\textbf{6p}{]} Write down an expression for the gradients of all
three weights: \$ \frac{∂E}{∂w(x,1)},
\frac{∂E}{∂w(1,2)},\frac{∂E}{∂w(1,3)}\$.

    Going backwards through the network,

\[ \newcommand{\partial}[1]{∂ #1} \]

\[ \frac{\partial{E}}{\partial{w(1,3)}} = \sum_{i \space \epsilon \space data } \frac{\partial{(z_2 − y_{2i})^2}}{\partial{w(1,3)}} + \frac{\partial{(z_3 − y_{3i})^2}}{\partial{w(1,3)}}\]

\[ \frac{\partial{E}}{\partial{w(1,3)}} = \sum_{i \space \epsilon \space data } \frac{\partial{(z_3 − y_{3i})^2}}{\partial{w(1,3)}}\]

\[ \frac{\partial{E}}{\partial{w(1,3)}} = \sum_{i \space \epsilon \space data } 2(z_3 − y_{3i}) \cdot \frac{\partial{z_3}}{\partial{w(1,3)}}\]

\[ \frac{\partial{E}}{\partial{w(1,3)}} = \sum_{i \space \epsilon \space data } 2(z_3 − y_{3i}) \cdot \frac{\partial{h(z_1 \cdot w_{(1,3)})}}{\partial{w(1,3)}}\]

\[ since \quad \quad \frac{\partial{h(x)}}{\partial(x)} = h(x) \cdot (1 - h(x)), \quad \quad \space <sigmoid>\]

\[ \frac{\partial{E}}{\partial{w(1,3)}} = \sum_{i \space \epsilon \space data } 2(z_3 − y_{3i}) \cdot h(z_1 \cdot w_{(1,3)}) \cdot (1-h(z_1 \cdot w_{(1,3)})) \cdot \frac{\partial{(z_1 \cdot w_{(1,3)})}}{\partial{w(1,3)}}\]

\[ \frac{\partial{E}}{\partial{w(1,3)}} = \sum_{i \space \epsilon \space data } 2(z_3 − y_{3i}) \cdot h(z_1 \cdot w_{(1,3)}) \cdot (1-h(z_1 \cdot w_{(1,3)})) \cdot z_1 \]

Likewise,

\[ \frac{\partial{E}}{\partial{w(1,2)}} = \sum_{i \space \epsilon \space data } 2(z_2 − y_{2i}) \cdot h(z_1 \cdot w_{(1,2)}) \cdot (1-h(z_1 \cdot w_{(1,2)})) \cdot z_1 \]

As for \(w_{(x,1)}\),

\[ \frac{\partial{E}}{\partial{w(x,1)}} = \sum_{i \space \epsilon \space data } \frac{\partial{(z_2 − y_{2i})^2}}{\partial{w(x,1)}} + \frac{\partial{(z_3 − y_{3i})^2}}{\partial{w(x,1)}}\]

\[ \frac{\partial{E}}{\partial{w(x,1)}} = \sum_{a \space \epsilon \space (2,3)} \sum_{i \space \epsilon \space data } \frac{\partial{(z_a − y_{ai})^2}}{\partial{w(x,1)}} \]

\[ \frac{\partial{E}}{\partial{w(x,1)}} = \sum_{a \space \epsilon \space (2,3)} \sum_{i \space \epsilon \space data } 2(z_a − y_{ai})\frac{\partial{(z_a − y_{ai})}}{\partial{w(x,1)}} \]

\[ \frac{\partial{E}}{\partial{w(x,1)}} = \sum_{a \space \epsilon \space (2,3)} \sum_{i \space \epsilon \space data } 2(z_a − y_{ai})\frac{\partial{(z_a)}}{\partial{w(x,1)}} \]

\[ \frac{\partial{E}}{\partial{w(x,1)}} = \sum_{a \space \epsilon \space (2,3)} \sum_{i \space \epsilon \space data } 2(z_a − y_{ai})\frac{\partial{(h(z_1 \cdot w_{(1,a)}))}}{\partial{w(x,1)}} \]

\[ \frac{\partial{E}}{\partial{w(x,1)}} = \sum_{a \space \epsilon \space (2,3)} \sum_{i \space \epsilon \space data } 2(z_a − y_{ai})(h(z_1 \cdot w_{(1,a)})(1-h(z_1 \cdot w_{(1,a)})\frac{\partial{(z_1 \cdot w_{(1,a)})}}{\partial{w(x,1)}} \]

\[ \frac{\partial{E}}{\partial{w(x,1)}} = \sum_{a \space \epsilon \space (2,3)} \sum_{i \space \epsilon \space data } 2(z_a − y_{ai})(z_a)(1-z_a)\frac{\partial{(z_1 \cdot w_{(1,a)})}}{\partial{w(x,1)}} \]

\[ \frac{\partial{E}}{\partial{w(x,1)}} = \sum_{a \space \epsilon \space (2,3)} \sum_{i \space \epsilon \space data } 2(z_a − y_{ai})(z_a)(1-z_a)(w_{(1,a)})\frac{\partial{(z_1)}}{\partial{w(x,1)}} \]

\[ \frac{\partial{E}}{\partial{w(x,1)}} = \sum_{a \space \epsilon \space (2,3)} \sum_{i \space \epsilon \space data } 2(z_a − y_{ai})(z_a)(1-z_a)(w_{(1,a)})\frac{\partial{h(x_i \cdot w_{(x,1)})}}{\partial{w(x,1)}} \]

\[ \frac{\partial{E}}{\partial{w(x,1)}} = \sum_{a \space \epsilon \space (2,3)} \sum_{i \space \epsilon \space data } 2(z_a − y_{ai})(z_a)(1-z_a)(w_{(1,a)})(h(x_i \cdot w_{(x,1)}))(1-h(x_i \cdot w_{(x,1)}))\frac{\partial{(x_i \cdot w_{(x,1)})}}{\partial{w(x,1)}} \]

\[ \frac{\partial{E}}{\partial{w(x,1)}} = \sum_{a \space \epsilon \space (2,3)} \sum_{i \space \epsilon \space data } 2(z_a − y_{ai})(z_a)(1-z_a)(w_{(1,a)})(z_1)(1-z_1)\frac{\partial{(x_i \cdot w_{(x,1)})}}{\partial{w(x,1)}} \]

\[ \frac{\partial{E}}{\partial{w(x,1)}} = \sum_{a \space \epsilon \space (2,3)} \sum_{i \space \epsilon \space data } 2(z_a − y_{ai})(z_a)(1-z_a)(w_{(1,a)})(z_1)(1-z_1) \cdot x_i \]

    \subsection{Coding Part}\label{coding-part}

    \begin{Verbatim}[commandchars=\\\{\}]
{\color{incolor}In [{\color{incolor}7}]:} \PY{k+kn}{import} \PY{n+nn}{torch}
        \PY{k+kn}{from} \PY{n+nn}{wk3\PYZus{}homework} \PY{k}{import} \PY{o}{*}
        \PY{k+kn}{import} \PY{n+nn}{matplotlib}\PY{n+nn}{.}\PY{n+nn}{pyplot} \PY{k}{as} \PY{n+nn}{plt}
        \PY{k+kn}{from} \PY{n+nn}{torchvision} \PY{k}{import} \PY{n}{transforms}
\end{Verbatim}


    \begin{Verbatim}[commandchars=\\\{\}]
{\color{incolor}In [{\color{incolor}8}]:} \PY{n}{dataset\PYZus{}count} \PY{o}{=} \PY{l+m+mi}{100}
        \PY{n}{root\PYZus{}path} \PY{o}{=} \PY{l+s+s1}{\PYZsq{}}\PY{l+s+s1}{../datasets/imagenet\PYZus{}first2500/}\PY{l+s+s1}{\PYZsq{}}
        \PY{n}{data\PYZus{}singlecrop} \PY{o}{=} \PY{n}{Wk3Dataset}\PY{p}{(}\PY{n}{root\PYZus{}path}\PY{p}{,} \PY{n}{data\PYZus{}limit}\PY{o}{=}\PY{n}{dataset\PYZus{}count}\PY{p}{,} \PY{n}{five\PYZus{}crop}\PY{o}{=}\PY{k+kc}{False}\PY{p}{)}
        
        \PY{c+c1}{\PYZsh{} testing Dataset subclass \PYZhy{} single center crop}
        \PY{n}{first\PYZus{}img} \PY{o}{=} \PY{n}{data\PYZus{}singlecrop}\PY{p}{[}\PY{l+m+mi}{0}\PY{p}{]}
        \PY{n+nb}{print}\PY{p}{(}\PY{n}{first\PYZus{}img}\PY{p}{[}\PY{l+s+s1}{\PYZsq{}}\PY{l+s+s1}{label}\PY{l+s+s1}{\PYZsq{}}\PY{p}{]}\PY{p}{,} \PY{n}{data\PYZus{}singlecrop}\PY{o}{.}\PY{n}{classes}\PY{p}{[}\PY{n}{first\PYZus{}img}\PY{p}{[}\PY{l+s+s1}{\PYZsq{}}\PY{l+s+s1}{label}\PY{l+s+s1}{\PYZsq{}}\PY{p}{]}\PY{p}{]}\PY{p}{)}
        \PY{n}{image} \PY{o}{=} \PY{n}{transforms}\PY{o}{.}\PY{n}{ToPILImage}\PY{p}{(}\PY{p}{)}\PY{p}{(}\PY{n}{first\PYZus{}img}\PY{p}{[}\PY{l+s+s1}{\PYZsq{}}\PY{l+s+s1}{image}\PY{l+s+s1}{\PYZsq{}}\PY{p}{]}\PY{p}{)}
        \PY{n}{plt}\PY{o}{.}\PY{n}{imshow}\PY{p}{(}\PY{n}{image}\PY{p}{)}
\end{Verbatim}


    \begin{Verbatim}[commandchars=\\\{\}]
65 sea snake

    \end{Verbatim}

\begin{Verbatim}[commandchars=\\\{\}]
{\color{outcolor}Out[{\color{outcolor}8}]:} <matplotlib.image.AxesImage at 0x1f6838951d0>
\end{Verbatim}
            
    \begin{center}
    \adjustimage{max size={0.9\linewidth}{0.9\paperheight}}{output_19_2.png}
    \end{center}
    { \hspace*{\fill} \\}
    
    \begin{Verbatim}[commandchars=\\\{\}]
{\color{incolor}In [{\color{incolor}9}]:} \PY{n}{dataset\PYZus{}count} \PY{o}{=} \PY{l+m+mi}{100}
        \PY{n}{root\PYZus{}path} \PY{o}{=} \PY{l+s+s1}{\PYZsq{}}\PY{l+s+s1}{../datasets/imagenet\PYZus{}first2500/}\PY{l+s+s1}{\PYZsq{}}
        \PY{n}{data\PYZus{}fivecrop} \PY{o}{=} \PY{n}{Wk3Dataset}\PY{p}{(}\PY{n}{root\PYZus{}path}\PY{p}{,} \PY{n}{data\PYZus{}limit}\PY{o}{=}\PY{n}{dataset\PYZus{}count}\PY{p}{,} \PY{n}{five\PYZus{}crop}\PY{o}{=}\PY{k+kc}{True}\PY{p}{)}
        
        \PY{c+c1}{\PYZsh{} testing Dataset subclass \PYZhy{} single center crop}
        \PY{n}{first\PYZus{}img} \PY{o}{=} \PY{n}{data\PYZus{}fivecrop}\PY{p}{[}\PY{l+m+mi}{3}\PY{p}{]}
        \PY{n+nb}{print}\PY{p}{(}\PY{n}{first\PYZus{}img}\PY{p}{[}\PY{l+s+s1}{\PYZsq{}}\PY{l+s+s1}{label}\PY{l+s+s1}{\PYZsq{}}\PY{p}{]}\PY{p}{,} \PY{n}{data\PYZus{}fivecrop}\PY{o}{.}\PY{n}{classes}\PY{p}{[}\PY{n}{first\PYZus{}img}\PY{p}{[}\PY{l+s+s1}{\PYZsq{}}\PY{l+s+s1}{label}\PY{l+s+s1}{\PYZsq{}}\PY{p}{]}\PY{p}{]}\PY{p}{)}
        
        \PY{n}{pos} \PY{o}{=} \PY{p}{[}\PY{l+s+s1}{\PYZsq{}}\PY{l+s+s1}{center}\PY{l+s+s1}{\PYZsq{}}\PY{p}{,} \PY{l+s+s1}{\PYZsq{}}\PY{l+s+s1}{topleft}\PY{l+s+s1}{\PYZsq{}}\PY{p}{,} \PY{l+s+s1}{\PYZsq{}}\PY{l+s+s1}{topright}\PY{l+s+s1}{\PYZsq{}}\PY{p}{,}\PY{l+s+s1}{\PYZsq{}}\PY{l+s+s1}{botleft}\PY{l+s+s1}{\PYZsq{}}\PY{p}{,} \PY{l+s+s1}{\PYZsq{}}\PY{l+s+s1}{botright}\PY{l+s+s1}{\PYZsq{}}\PY{p}{]}
        \PY{k}{for} \PY{n}{i} \PY{o+ow}{in} \PY{n+nb}{range}\PY{p}{(}\PY{l+m+mi}{5}\PY{p}{)}\PY{p}{:}
            \PY{n}{image} \PY{o}{=} \PY{n}{transforms}\PY{o}{.}\PY{n}{ToPILImage}\PY{p}{(}\PY{p}{)}\PY{p}{(}\PY{n}{first\PYZus{}img}\PY{p}{[}\PY{l+s+s1}{\PYZsq{}}\PY{l+s+s1}{image}\PY{l+s+s1}{\PYZsq{}}\PY{p}{]}\PY{p}{[}\PY{n}{i}\PY{p}{]}\PY{p}{)}
            \PY{n}{plt}\PY{o}{.}\PY{n}{figure}\PY{p}{(}\PY{p}{)}
            \PY{n}{plt}\PY{o}{.}\PY{n}{title}\PY{p}{(}\PY{n}{pos}\PY{p}{[}\PY{n}{i}\PY{p}{]}\PY{p}{)}
            \PY{n}{plt}\PY{o}{.}\PY{n}{imshow}\PY{p}{(}\PY{n}{image}\PY{p}{)}
\end{Verbatim}


    \begin{Verbatim}[commandchars=\\\{\}]
809 soup bowl

    \end{Verbatim}

    \begin{center}
    \adjustimage{max size={0.9\linewidth}{0.9\paperheight}}{output_20_1.png}
    \end{center}
    { \hspace*{\fill} \\}
    
    \begin{center}
    \adjustimage{max size={0.9\linewidth}{0.9\paperheight}}{output_20_2.png}
    \end{center}
    { \hspace*{\fill} \\}
    
    \begin{center}
    \adjustimage{max size={0.9\linewidth}{0.9\paperheight}}{output_20_3.png}
    \end{center}
    { \hspace*{\fill} \\}
    
    \begin{center}
    \adjustimage{max size={0.9\linewidth}{0.9\paperheight}}{output_20_4.png}
    \end{center}
    { \hspace*{\fill} \\}
    
    \begin{center}
    \adjustimage{max size={0.9\linewidth}{0.9\paperheight}}{output_20_5.png}
    \end{center}
    { \hspace*{\fill} \\}
    
    \begin{Verbatim}[commandchars=\\\{\}]
{\color{incolor}In [{\color{incolor}10}]:} \PY{c+c1}{\PYZsh{} running validation with normal centercrop}
         \PY{n}{valset}\PY{p}{,} \PY{n}{model} \PY{o}{=} \PY{n}{run\PYZus{}validation}\PY{p}{(}\PY{n}{five\PYZus{}crop}\PY{o}{=}\PY{k+kc}{False}\PY{p}{,} \PY{n}{dataset\PYZus{}count}\PY{o}{=}\PY{l+m+mi}{250}\PY{p}{)}
\end{Verbatim}


    \begin{Verbatim}[commandchars=\\\{\}]
dataset length 250 torch.Size([3, 224, 224])
Val - Epoch 0..
      >> Epoch loss 0.33772 accuracy 0.664                      in 8.3146s

    \end{Verbatim}

    \begin{Verbatim}[commandchars=\\\{\}]
{\color{incolor}In [{\color{incolor}11}]:} \PY{c+c1}{\PYZsh{} running validation with fivecrop}
         \PY{n}{valset}\PY{p}{,} \PY{n}{model} \PY{o}{=} \PY{n}{run\PYZus{}validation}\PY{p}{(}\PY{n}{five\PYZus{}crop}\PY{o}{=}\PY{k+kc}{True}\PY{p}{,} \PY{n}{dataset\PYZus{}count}\PY{o}{=}\PY{l+m+mi}{250}\PY{p}{)}
\end{Verbatim}


    \begin{Verbatim}[commandchars=\\\{\}]
dataset length 250 torch.Size([5, 3, 224, 224])
Val - Epoch 0..
      >> Epoch loss 0.31602 accuracy 0.692                      in 21.3460s

    \end{Verbatim}


    % Add a bibliography block to the postdoc
    
    
    
    \end{document}
